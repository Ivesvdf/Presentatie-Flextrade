\documentclass[pdflatex,compress,usenames, dvipsnames]{beamer}

%\usetheme[darktitle,framenumber,totalframenumber]{UniversiteitAntwerpen}
\usetheme[dark,framenumber,totalframenumber,notes=hide]{UniversiteitAntwerpen}
% \setbeamertemplate{background}[grid][step=1cm]
% \beamertemplategridbackground{1}

\usepackage{pgfpages}

\usepackage{lipsum}
\usepackage{colortbl}
\usepackage{lmodern}
\usepackage{booktabs}
\usepackage{listings}
\usepackage{bm}
\usepackage{fancyvrb}
\usepackage{graphicx}
\usepackage{tikz}
\usepackage{wrapfig}
\usepackage{textpos}
\usepackage[english]{babel}

% the official UA font:
\IfFileExists{Auto1.sty}{
    \usepackage[lf]{Auto1}
}

\usepackage{mathptmx}

\title{Flextrade}

\author{Ives~van~der~Flaas \\
Nathan Samson}

%\renewcommand{\emph}{\textbf}
% The following corrects spacing! There was too much space under the title and
% this fixes it. 
\let\OldFrameTitle\frametitle
\renewcommand{\frametitle}[1]{\OldFrameTitle{#1} \vspace{-1cm}}

% Use instead of \alert which looks like shit outside of blocks\ldots
\newcommand{\coloremph}[1]{{\color{lime} #1}}
\newcommand\startsection[1]{\section{#1} \frame{\begin{center}\Huge #1
\end{center}}}

\setbeamerfont{section in toc}{size=\Large}
\setbeamercolor{subsection in toc}{fg=green!90!black, bg=blue}
\setbeamercolor{section in toc}{fg=uablue!30!white, bg=blue}
\setbeamercolor{section in toc shaded}{fg=uablue!80!white,bg=blue}
\setbeamercolor{subsection in toc shaded}{fg=uablue!80!white,bg=blue}
\setbeamertemplate{section in toc}[sections numbered]
\setbeamertemplate{section in toc shaded}[default][90]
\setbeamertemplate{subsection in toc shaded}[default][90]

\newcommand\tocpage{
\begin{frame}
	\frametitle{Table of Contents}
	\tableofcontents[sectionstyle=show/shaded,subsectionstyle=show/shaded]
\end{frame}
}

\newcommand{\imgslide}[1]{{
	\usebackgroundtemplate{\includegraphics[width=\paperwidth]{#1}}
	\begin{frame}[plain]
	\end{frame}}
}

\begin{document}

% ----------------------------------------------------------------------------
% *** Titlepage <<<
% ----------------------------------------------------------------------------
\maketitle
% ----------------------------------------------------------------------------
% *** END of Titlepage >>>
% ----------------------------------------------------------------------------
\definecolor{bg}{RGB}{58,94,133}%

%\frame{\partpage}
\section*{Architecture}
\begin{frame}
	\frametitle{Bank Architecture}
	\begin{center}
	\begin{tikzpicture}
		%Arrows interbank
		\draw[<->] (1.4, -2.5) -- (6,0.5);
		\draw[<->] (3, -2.5) -- (6,0.5);

		\draw[<->] (3, 0.5) -- (6, -2.5);
		\draw[<->] (3, 0.5) -- (6+1.6, -2.5);
		%Bank 1
		\draw[very thick,fill=uablue] (0,0) rectangle (3,1);
		\node[thick] at (3/2,1/2){Bank};

		% JS 1
		\draw[fill=bg] (0,-1.5) rectangle +(3,1);
		\node[] at (3/2,-1.5+1/2){Javaspace};
		\draw[<->] (3/2, 0) -- (3/2, -1.5+1);

		%workers
		\draw (0,-3) rectangle +(1.4, 1);
		\node[font=\tiny,text width=1.4, align=center] at (0.2, -2.5) { Transaction\-Worker };
		\draw[fill=bg] (1.6,-3) rectangle +(1.4, 1);
		\node[font=\tiny,text width=1.4, align=center] at (0.2+1.6, -2.5) {
		Preparing\-Transaction\-Worker };

		\draw[<->] (1.4/2, -1.5) -- +(0, -0.5);
		\draw[<->] (1.6+1.4/2, -1.5) -- +(0, -0.5);


		%Bank 2
		\draw[very thick,fill=uablue] (6,0) rectangle +(3,1);
		\node[thick] at (6+3/2,1/2){Bank};
		
		% JS 2
		\draw[fill=bg] (6,-1.5) rectangle +(3,1);
		\node[] at (6+3/2,-1.5+1/2){Javaspace};
		\draw[<->] (6+3/2, 0) -- ++(0, -1.5+1);

		%
		%workers
		\draw[fill=bg] (6+0,-3) rectangle +(1.4, 1);
		\node[font=\tiny,text width=1.4, align=center] at (6+0.2, -2.5) { Transaction\-Worker };
		\draw (6+1.6,-3) rectangle +(1.4, 1);
		\node[font=\tiny,text width=1.4, align=center] at (6+0.2+1.6, -2.5) {
		Preparing\-Transaction\-Worker };

		\draw[<->] (6+1.4/2, -1.5) -- +(0, -0.5);
		\draw[<->] (6+1.6+1.4/2, -1.5) -- +(0, -0.5);


	\end{tikzpicture}
	\end{center}
\end{frame}

\begin{frame}
	\frametitle{Auction Architecture}
	\begin{center}
	\begin{tikzpicture}[scale=0.8]
		\draw[very thick,fill=uablue] (-1.5, 0) rectangle +(3, 1); 
		\node at (0,0.5){Auction };

		\draw (-1.5, -1.5) rectangle +(3,1);
		\node at (0, -1) { JavaSpace };
		\draw[<->] (0, 0) -- +(0,-0.5);

		\draw (-3.2, -3.5) rectangle +(3,1);
		\node[font=\footnotesize,text width=1,align=center] at (-3.0, -3) { Auctioneer\-Manager };
		\draw (0.2, -3.5) rectangle +(3,1);
		\node[font=\footnotesize,text width=1,align=center] at (0.4, -3) { Auctioneer\-Manager };
		\draw[<->] (0,-1.5) -- +(-1.5, -1);
		\draw[<->] (0,-1.5) -- +(+1.5, -1);

		\draw (-3.2,-5) rectangle +(3,1);
		\node[font=\footnotesize] at (-1.7,-5+0.5) { ShippingService };

		\draw (0.2,-5) rectangle +(3,1);
		\node[font=\footnotesize] at (+1.7,-5+0.5) { Bank };

		\draw[->] (-1.7, -3.5) -- +(0, -0.5);
		\draw[->] (-1.7, -3.5) -- +(3.2, -0.5);
		\draw[->] (+1.7, -3.5) -- +(0, -0.5);
		\draw[->] (+1.7, -3.5) -- +(-3.2, -0.5);

		\draw (-5, 0) rectangle +(3,1);
		\node[font=\footnotesize] at (-3.5, 0.5) { Webservice };
		\draw[->] (-2, 0.5) -- (-1.5, 0.5);

	\end{tikzpicture}
	\end{center}
\end{frame}

\begin{frame}
	\frametitle{Responsibilities}
	\begin{center}
		% AANVULLEN!
		\begin{tabular}{ll}
			\toprule
			Ives & Nathan \\
			\midrule
			JavaSpace worker & \ldots \\
	 		GUI & \\
			\bottomrule
		\end{tabular}
	\end{center}
\end{frame}

\end{document}
\startsection{Virtualization}
\begin{frame}
	\frametitle{What is virtualization?}
\begin{block}{Definition: Wikipedia}
The creation of a virtual (rather than actual) version of something, such as a
hardware platform, operating system, \ldots
\end{block}
	\begin{itemize}
	  \item May provide better security without hampering performance
	  \item Provides isolation between instances/platforms
	\end{itemize}
\end{frame}
\subsection{OS-level}
\begin{frame}
	\frametitle{Virtual machines}
	\begin{itemize}
	  \item Virtual Machine Monitor provides virtualization of hardware
	  \item Virtual Machines execute on this virtual system
	  \item Each VM is an operating system instance with own environment
	\end{itemize}
\end{frame}
\begin{frame}
	\frametitle{Virtual machines}
	\begin{itemize}
	  \item Early breakthrough: \emph{Full system virtualization}
	  \begin{itemize}
	    \item Exact copy of machine hardware
	    \item VMWare, VirtualBox, \ldots
	  \end{itemize}
	  \item Recent development: \emph{Paravirtualization}
	  \begin{itemize}
	    \item Non-exact abstraction of hardware
	    \item Denali, Xen, \ldots
	  \end{itemize}
	\end{itemize}
\end{frame}
\begin{frame}
	\frametitle{Example: relocation}
	\begin{figure}
	  \centering
	  %	  \includegraphics[width=\paperwidth/2]{q3.png}
	\end{figure}
\end{frame}
\begin{frame}
	\frametitle{Example: relocation}
	\begin{itemize}
	  \item Take a running VM, hosting a multiplayer match
	  \item Copy it to another server
	  \item Copy the memory step by step
	  \item Shut down and resume on the new server
	  \item ... how much downtime would occur?
	\end{itemize}
\end{frame}
\begin{frame}
	\frametitle{Example: relocation}
	\begin{itemize}
	  \item Just 50ms of downtime
	  \item None of the players noticed
	\end{itemize}
\end{frame}
\begin{frame}
	\frametitle{Virtual machines}
	\begin{itemize}
	  \item Early breakthrough: \emph{Full system virtualization}
	  \begin{itemize}
	    \item Exact copy of machine hardware
	    \item VMWare, VirtualBox, \ldots
	  \end{itemize}
	  \item Recent development: \emph{Paravirtualization}
	  \begin{itemize}
	    \item Non-exact abstraction of hardware
	    \item Denali, Xen, \ldots
	  \end{itemize}
	\end{itemize}
\end{frame}
\begin{frame}
	\frametitle{Full system virtualization}
	\begin{itemize}
	  \item Advantages:
	  \begin{itemize}
	    \item Operating systems can run unmodified
	    \item Programs will behave the same on virtual machines
	  \end{itemize}
	  \item Disadvantages:
	  \begin{itemize}
	    \item Most architectures are not made for virtualization ...
	    \item Which means the VMM needs to emulate certain instructions ...
	    \item Leading to decreased performance
	  \end{itemize}
	\end{itemize}
\end{frame}
\begin{frame}
	\frametitle{Paravirtualization}
	\begin{itemize}
	  \item The idea: collaboration between (adapted) OS and VMM
	  \item Provides an abstraction of the platform, not a copy
	  \item Less overhead, better resource allocation
	\end{itemize}
\end{frame}
\subsection{Application-level}
\startsection{Security Impact}
\subsection{OS-level}
\subsection{Application-level}

\begin{frame}
	\centering
	\Huge Questions? 

\end{frame}












%%%%%%%%%%%%%%%%%%%%%%%%%%%%%%%%%%%%%%%%%%%%%%%%%%%%%%%%%%%%%%%%%%%%%%%%%%
%%%%%%%%%%%%%%%%%%%%%%%%%%%%%%%%%%%%%%%%%%%%%%%%%%%%%%%%%%%%%%%%%%%%%%%%%%
%%%%%%%%%%%%%%%%%%%%%%%%%%%%%%%%%%%%%%%%%%%%%%%%%%%%%%%%%%%%%%%%%%%%%%%%%%
%%%%%%%%%%%%%%%%%%%%%%%%%%%%%%%%%%% EXAMPLES %%%%%%%%%%%%%%%%%%%%%%%%%%%%%
%%%%%%%%%%%%%%%%%%%%%%%%%%%%%%%%%%%%%%%%%%%%%%%%%%%%%%%%%%%%%%%%%%%%%%%%%%
%%%%%%%%%%%%%%%%%%%%%%%%%%%%%%%%%%%%%%%%%%%%%%%%%%%%%%%%%%%%%%%%%%%%%%%%%%
%%%%%%%%%%%%%%%%%%%%%%%%%%%%%%%%%%%%%%%%%%%%%%%%%%%%%%%%%%%%%%%%%%%%%%%%%%
\begin{frame}
\frametitle{How?}
By using a (discrete) \emph{Fourier Transform}!
\begin{enumerate}
	\item Use Fourier Transform to convert from the spacial domain to the frequency domain
	\item Apply filter in the frequency domain
	\item Use Inverse Fourier Transform to convert back to the spacial domain
\end{enumerate}
\end{frame}

\begin{frame}
	\frametitle{Jean Baptiste Joseph Fourier}
	\note[item]{Legt Fourier analyse terloops uit in La Th\'eorie Analitique
	de la Chaleur, 1822 analoog aan Newton die Calculus uitvond omdat hij het
	nodig had.}
	\begin{block}{Contribution: Fourier Analysis}
		Every periodic function can be represented as a sum of sines or
		cosines (some restrictions apply). 
	\end{block}

	\begin{block}{Contribution: Fourier Transform}
		When functions aren't periodic, but have a finite area under
		their curve, they can still be represented as an \alert{integral} of sines
		or cosines, multiplied by some function. 
	\end{block}
\end{frame}



% subfloat example
\begin{frame}
	\frametitle{Example: Spectrum}
	\begin{figure}
		\centering
		\subfloat[Original Image]{\includegraphics[width=0.3\textwidth]{blockspacial}}
		\quad
		\subfloat[Fourier Spectrum]{\includegraphics[width=0.3\textwidth]{blockfrequency}}
	\end{figure}
	{\tiny Image Source:
	http://www.cs.utah.edu/$\sim{}$seyung/index\_files/projects/cs7966/project3/index.html}
\end{frame}



















% ----------------------------------------------------------------------------
% *** Test frame <<<
% ----------------------------------------------------------------------------
\begin{frame}
\frametitle{A first test frame}
\lipsum[1]
\end{frame}
% ----------------------------------------------------------------------------
% *** END of Test frame >>>
% ----------------------------------------------------------------------------

% ----------------------------------------------------------------------------
% *** Test frame with overflow <<<
% ----------------------------------------------------------------------------
\begin{frame}
\frametitle{Test frame with overflow}
\lipsum
\end{frame}
% ----------------------------------------------------------------------------
% *** END of Test frame >>>
% ----------------------------------------------------------------------------


% ----------------------------------------------------------------------------
% *** Test frame with overflow <<<
% ----------------------------------------------------------------------------
%\begin{frame}
%\frametitle{The is a test frame with a pretty long frame title}
%\lipsum
%\end{frame}
% ----------------------------------------------------------------------------
% *** END of Test frame >>>
% ----------------------------------------------------------------------------

\subsection{My subsection 2}
% ----------------------------------------------------------------------------
% *** Test frame with Itemize <<<
% ----------------------------------------------------------------------------
\begin{frame}
\frametitle{Test frame with itemize}

\begin{itemize}
    \item<1-> firstly
    \item<2-> secondly
        \begin{itemize}
            \item sub-item
            \item another sub-item
        \end{itemize}
      \item<3-> thirdly
\end{itemize}

\end{frame}
% ----------------------------------------------------------------------------
% *** END of Test frame with Itemize >>>
% ----------------------------------------------------------------------------


% ----------------------------------------------------------------------------
% *** Test frame with Math <<<
% ----------------------------------------------------------------------------
\begin{frame}
\frametitle{A math frame}

\begin{theorem}[Pythagoras]
The square of the hypotenuse of a \alert{right} triangle is equal to the sum of the squares on the other two sides:
\[
a^2 + b^2 = c^2.
\]
\end{theorem}
\begin{proof}
Straightforward.
\end{proof}

\end{frame}
% ----------------------------------------------------------------------------
% *** END of Test frame with Math >>>
% ----------------------------------------------------------------------------


% ----------------------------------------------------------------------------
% *** Test frame with Environments <<<
% ----------------------------------------------------------------------------
\begin{frame}
\frametitle{Environments}

\begin{definition}
A \emph{prime number} (or a prime) is a natural number which has exactly two distinct natural number divisors: 1 and itself. 
\end{definition}

\begin{exampleblock}{Example}
The first five prime numbers are $2$, $3$, $5$, $7$, and $11$.
\end{exampleblock}

\begin{alertblock}{Alert block}
Note that $1$ is not a prime number.
\end{alertblock}

\end{frame}
% ----------------------------------------------------------------------------
% *** END of Test frame with Environments >>>
% ----------------------------------------------------------------------------


